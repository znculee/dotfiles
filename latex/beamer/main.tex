%! TEX program = pdflatex

\documentclass[
%aspectratio=169, % 1610, 169, 149, 141, 54, 43, 32
%handout,
]{beamer}

\usetheme[
%dark, % dark mode
%grid, % show grid
%note, % show notes
outline, % show running outlines after \section
]{xtli}

\usepackage{pgfplots}
\usepackage{pgfplotstable}
\pgfplotsset{compat=1.16}

\title{There Is No Largest Prime Number}
\subtitle{Subtitle}
\author{Euclid of Alexandria}
\date{27th International Symposium of Prime Numbers}

\begin{document}

\begin{frame}[noframenumbering,plain]
\titlepage
\end{frame}

\begin{frame}[noframenumbering,plain]
\frametitle{Outline}
\tableofcontents
\end{frame}

\section{Motivation}
\subsection{The Basic Problem That We Studied}

\begin{frame}
\frametitle{What Are Prime Numbers?}
\begin{definition}
  A \alert{prime number} is a number that has exactly two divisors.
\end{definition}
\begin{example}
  \begin{itemize}
    \item 2 is prime (two divisors: 1 and 2).
      \pause
    \item 3 is prime (two divisors: 1 and 3).
      \pause
    \item 4 is not prime (\alert{three} divisors: 1, 2, and 4).
  \end{itemize}
\end{example}
\markerlessfootnote{More primer numbers: 2, 3, 5, 7, 11, 13, 17, 19, 23, 29, 31, 37, 41, 43, 47, 53, ...}
\end{frame}

\begin{frame}[t]
\frametitle{There Is No Largest Prime Number}
\framesubtitle{The proof uses \textit{reductio ad absurdum}.}
\begin{theorem}[c. 300 BC]
  There is no largest prime number.~\footnote{This is a footnote.}
\end{theorem}
\begin{proof}
  \begin{enumerate}
    \item<1-> Suppose $p$ were the largest prime number.
    \item<2-> Let $q$ be the product of the first $p$ numbers.
    \item<3-> Then $q + 1$ is not divisible by any of them.
    \item<1-> But $q + 1$ is greater than $1$, thus divisible by some prime
      number not in the first $p$ numbers.\qedhere
  \end{enumerate}
\end{proof}
\uncover<4->{The proof used \textit{reductio ad absurdum}.}
\end{frame}

\begin{frame}
\frametitle{What’s Still To Do?}
\begin{columns}[t]
  \column{.5\textwidth}
  \begin{block}{Answered Questions}
    How many primes are there?
    \footnote[frame]{A test footnote in the first column.}
  \end{block}
  \pause
  \column{.5\textwidth}
  \begin{block}{Open Questions}
    Is every even number the sum of two primes?
    \footnote[frame]{A test footnote in the second column.}
  \end{block}
\end{columns}
\end{frame}

\begin{frame}[fragile]
\frametitle{An Algorithm For Finding Prime Numbers.}
\begin{semiverbatim}
int main (void) \{
  \alert<2>{std::}vector<bool> is_prime (100, true);
      for (int i = 2; i < 100; i++)
        if (is_prime[i]) \{
          \alert<2>{std::}cout << i << " ";
          for (int j = i; j < 100;
            is_prime [j] = false, j+=i);
        \}
  return 0;
\}
\end{semiverbatim}
\visible<2>{
  \begin{alertblock}{}
    Note the use of \alert{\texttt{std::}}.
  \end{alertblock}
}
\end{frame}

\section{Second Section}

\begin{frame}
\frametitle{List}
\pdfnote{%
My presentation notes.\textCR
Notes in the second line.
}
\begin{columns}[t]
\column{.5\textwidth}
\begin{itemize}
  \item XXX
  \begin{itemize}
    \item XXX
    \item XXX
      \begin{itemize}
        \item XXX
      \end{itemize}
  \end{itemize}
  \item XXX~\cite{vaswani2017attention}
  \item XXX
\end{itemize}
\column{.5\textwidth}
\begin{enumerate}
  \item XXX
  \begin{enumerate}
    \item XXX
    \item XXX
      \begin{enumerate}
        \item XXX
      \end{enumerate}
  \end{enumerate}
  \item XXX
  \item XXX
\end{enumerate}
\end{columns}
\begin{tikzpicture}[remember picture, overlay, shift={(current page.center)}]
  \tikzstyle{Box} = [rectangle, rounded corners, align=center, draw=c4fg];
  \tikzstyle{Callout} = [ellipse callout, draw=c4fg];
  \tikzstyle{Arrow} = [-{Latex[scale=1]}, thick];
  \node[Box] at (0,0) (org) {position\\(0,0)};
  \node[Box] at (5,3) {position\\(5,3)};
  \node[Callout, callout absolute pointer={($(org)+(-0.8,-0.8)$)}, position=225:2 from org] {polar coordinates};
  \node[Callout, callout absolute pointer={(-5.1,1.2)}] at (-4,2) {itemize};
  \node[Callout, callout absolute pointer={(0.9,1.2)}] at (2,2) {enumerate};
  \draw[Arrow] (2,-3) to node[pos=0.5, above, sloped] {Arrow} (4,-2);
\end{tikzpicture}
\end{frame}

\begin{frame}
\frametitle{Pictures with TikZ~\cite{tantau2008tikz}}
\begin{center}
\begin{tikzpicture}[
  scale=1.5,
  dot/.style={circle,draw=c4fg,fill=c4fg,thick,inner sep=0pt,minimum size=2pt},
  background rectangle/.style={fill=c4bg},
  show background rectangle
]
  \draw[c4fg!30!c4bg] (-1.5,0) -- (1.5,0);
  \draw[c4fg!30!c4bg] (0,-1.5) -- (0,1.5);
  \node[dot] at (-1,0) (n1) {};
  \node[dot] at (0,1)  (n2) {};
  \node[dot] at (1,0)  (n3) {};
  \node[dot] at (0,-1) (n4) {};
  \draw[c4fg,thick] (n1) -- (n2) -- (n3) -- (n4) -- (n1) -- cycle;
  \draw[c4al,thick] (-1,0.5) -- (0,1) -- (1,1.5);
\end{tikzpicture}
\qquad
\begin{tikzpicture}[
  scale=1.5,
  background rectangle/.style={fill=c4bg},
  show background rectangle
]
  \draw[c4fg!30!c4bg] (-1.5,0) -- (1.5,0);
  \draw[c4fg!30!c4bg] (0,-1.5) -- (0,1.5);
  \draw[c4fg,thick] (-1,1) -- (0,0) -- (1,1);
\end{tikzpicture}
\end{center}
\end{frame}

\begin{frame}
\frametitle{Bar Chart with TikZ}
\begin{center}
\begin{tikzpicture}
    \begin{axis}[
        ybar,
        enlarge x limits=0.15,
        ymin=50,
        ymax=100,
        legend style={
            at={(0.5,-0.15)},
            anchor=north,
            legend columns=-1,
            draw=none,
            fill=c4bg,
        },
        axis line style={draw=none},
        tick style={draw=none},
        ymajorgrids,
        symbolic x coords={TreeAcc,Gram,Corr,Disc},
        xtick={TreeAcc,Gram,Corr,Disc},
        width=\textwidth,
        height=0.618\textwidth,
    ]
        \addplot[fill=c4st!30!c4bg] coordinates {(TreeAcc,nan) (Gram,98.77) (Corr,77.09) (Disc,79.04)};
        \addplot[fill=c4fg!30!c4bg] coordinates {(TreeAcc,nan) (Gram,96.7) (Corr,81.56) (Disc,83.93)};
        \addplot[fill=c4al!30!c4bg] coordinates {(TreeAcc,92.5) (Gram,95.26) (Corr,87.61) (Disc,85.97)};
        \addplot[fill=c4eg!30!c4bg] coordinates {(TreeAcc,96.92) (Gram,95.30) (Corr,91.82) (Disc,93.44)};
        \legend{\textsc{Flat},\textsc{Token},\textsc{Tree},\textsc{Constr}}
    \end{axis}
\end{tikzpicture}
\end{center}
\visible<2>{
\begin{tikzpicture}[remember picture, overlay, shift={(current page.center)}]
    \tikzstyle{Starburst} = [starburst, fill=c4st!10!c4bg, fill opacity=0.9, draw=black!100, thick, draw opacity=0.9, text width=5cm, minimum width=5cm, minimum height=3.09cm, text opacity=1, align=center];
    \node[Starburst, at={(0,0)}] {Tree based repersentations\\improve correctness};
    \draw [decorate, decoration={brace, amplitude=5pt}, very thick] (3.9,2.6) -- (4.8,2.6);
    \node[star, star points=5, at={(2.25,2.6)}, star point ratio=0.45, fill=c4al, inner sep=0.08cm] {};
\end{tikzpicture}
}
\end{frame}

\begin{frame}[fragile]
\frametitle{Line Chart with TikZ}
\pgfplotstableread{
    label         lbl   st-van st-cd st-rmr
    {253\\\%1}     19.58 27.56  89.97 38.00
    {507\\\%2}     41.65 49.31  91.32 61.33
    {1269\\\%5}    65.20 74.56  93.30 82.15
    {2539\\\%10}   80.29 84.59  95.03 88.82
    {5078\\\%20}   87.50 91.54  95.32 92.82
    {12695\\\%50}  91.96 94.52  96.64 95.35
    {25390\\\%100} 94.20 95.67  96.64 95.83
}\data
\begin{figure}
\begin{tikzpicture}
    \begin{axis}[
        tick label style={font=\tiny},
        xtick=data,
        xticklabels from table={\data}{label},
        xticklabel style={align=center},
        ytick distance=20,
        label style={font=\scriptsize},
        xlabel={\#Training Samples},
        ylabel={Tree Accuracy},
        legend style={
            font=\tiny,
            draw=none,
            fill=c4bg,
        },
        legend cell align={left},
        legend pos={south east},
        legend entries={
            \textsc{LBL},
            \textsc{ST-VAN},
            \textsc{ST-RMR},
            \textsc{ST-CD},
        },
        width=\textwidth,
        height=0.618\textwidth,
    ]
    \addplot [draw=c4st, mark=o, thick] table [meta=label, x expr=\coordindex, y=lbl] {\data};
    \addplot [draw=c4fg, mark=square, thick] table [meta=label, x expr=\coordindex, y=st-van] {\data};
    \addplot [draw=c4al, mark=diamond, thick] table [meta=label, x expr=\coordindex, y=st-rmr] {\data};
    \addplot [draw=c4eg, mark=triangle, thick] table [meta=label, x expr=\coordindex, y=st-cd] {\data};
    \draw[draw=gray, dashed, ultra thin] (axis cs:\pgfkeysvalueof{/pgfplots/xmin}, 94.20) -- (axis cs:\pgfkeysvalueof{/pgfplots/xmax}, 94.20);
    \end{axis}
\end{tikzpicture}
\end{figure}
\end{frame}

\begin{frame}
\frametitle{Wide Table in \alert{adjustwidth} Environment}
\begin{adjustwidth}{-2cm}{-2cm}
    \begin{table}
        \begin{tabular}{|c|c|c|c|c|} \hline
            SSSSSSS & AAAAAAA & TTTTTTT & OOOOOOO & RRRRRRR\\ \hline
            AAAAAAA & RRRRRRR & EEEEEEE & PPPPPPP & OOOOOOO\\ \hline
            TTTTTTT & EEEEEEE & NNNNNNN & EEEEEEE & TTTTTTT\\ \hline
            OOOOOOO & PPPPPPP & EEEEEEE & RRRRRRR & AAAAAAA\\ \hline
            RRRRRRR & OOOOOOO & TTTTTTT & AAAAAAA & SSSSSSS\\ \hline
        \end{tabular}
    \end{table}
\end{adjustwidth}
\end{frame}

\section*{References}

\begin{frame}[noframenumbering,plain]%[allowframebreaks]
\frametitle{References}
\footnotesize
\bibliographystyle{amsalpha}
\bibliography{ref.bib}
\end{frame}

\appendix

\section*{Convex Optimization}

\begin{frame}[t]
\frametitle{Karush–Kuhn–Tucker Conditions}
Nonlinear optimization problem\par\pause
Necessary conditions
\end{frame}

\section*{Matrix Analysis}

\begin{frame}
\frametitle{Singular Value Decomposition}
Statement of the theorem\par\pause
\vspace{\baselineskip}
Intuitive interpretations
\end{frame}

\end{document}
